% The first thing is to tell LaTeX that this is a document in the MBIE word format
\documentclass{mbie-word}\usepackage[]{graphicx}\usepackage[]{color}
%% maxwidth is the original width if it is less than linewidth
%% otherwise use linewidth (to make sure the graphics do not exceed the margin)
\makeatletter
\def\maxwidth{ %
  \ifdim\Gin@nat@width>\linewidth
    \linewidth
  \else
    \Gin@nat@width
  \fi
}
\makeatother

\definecolor{fgcolor}{rgb}{0.345, 0.345, 0.345}
\newcommand{\hlnum}[1]{\textcolor[rgb]{0.686,0.059,0.569}{#1}}%
\newcommand{\hlstr}[1]{\textcolor[rgb]{0.192,0.494,0.8}{#1}}%
\newcommand{\hlcom}[1]{\textcolor[rgb]{0.678,0.584,0.686}{\textit{#1}}}%
\newcommand{\hlopt}[1]{\textcolor[rgb]{0,0,0}{#1}}%
\newcommand{\hlstd}[1]{\textcolor[rgb]{0.345,0.345,0.345}{#1}}%
\newcommand{\hlkwa}[1]{\textcolor[rgb]{0.161,0.373,0.58}{\textbf{#1}}}%
\newcommand{\hlkwb}[1]{\textcolor[rgb]{0.69,0.353,0.396}{#1}}%
\newcommand{\hlkwc}[1]{\textcolor[rgb]{0.333,0.667,0.333}{#1}}%
\newcommand{\hlkwd}[1]{\textcolor[rgb]{0.737,0.353,0.396}{\textbf{#1}}}%

\usepackage{framed}
\makeatletter
\newenvironment{kframe}{%
 \def\at@end@of@kframe{}%
 \ifinner\ifhmode%
  \def\at@end@of@kframe{\end{minipage}}%
  \begin{minipage}{\columnwidth}%
 \fi\fi%
 \def\FrameCommand##1{\hskip\@totalleftmargin \hskip-\fboxsep
 \colorbox{shadecolor}{##1}\hskip-\fboxsep
     % There is no \\@totalrightmargin, so:
     \hskip-\linewidth \hskip-\@totalleftmargin \hskip\columnwidth}%
 \MakeFramed {\advance\hsize-\width
   \@totalleftmargin\z@ \linewidth\hsize
   \@setminipage}}%
 {\par\unskip\endMakeFramed%
 \at@end@of@kframe}
\makeatother

\definecolor{shadecolor}{rgb}{.97, .97, .97}
\definecolor{messagecolor}{rgb}{0, 0, 0}
\definecolor{warningcolor}{rgb}{1, 0, 1}
\definecolor{errorcolor}{rgb}{1, 0, 0}
\newenvironment{knitrout}{}{} % an empty environment to be redefined in TeX

\usepackage{alltt}
\usepackage{float}

\title{Title for word}

%The document begins here
\IfFileExists{upquote.sty}{\usepackage{upquote}}{}
\begin{document}

\maketitle

\newpage
\ClearShipoutPicture

\tableofcontents

\section{Section 1}
Your text goes here

\section{Section 2}
Your text goes here

\cite{Hu.1997}

\paragraph{para}
Text

\section{Another Section}

With text!
\section{Tables and figures}

\subsection{Table}

% latex table generated in R 3.0.1 by xtable 1.7-1 package
% Thu Feb 26 14:45:12 2015
\begin{tabular}[t]{|p{2.5cm}|p{1cm}|p{1.5cm}|p{1.5cm}|}
  \hline
Key International Markets & Market Share & No\_of\_Visit & Growth(pa) \\ 
  \hline
Australia & 43\% & 1,246,000 & 3\% \\ 
  China & 9\% &   258,000 & 11\% \\ 
  USA & 8\% &   219,000 & 11\% \\ 
  UK & 7\% &   194,000 & 2\% \\ 
  Japan & 3\% &    80,000 & 7\% \\ 
  Germany & 3\% &    77,000 & 13\% \\ 
   \hline
\multicolumn{4}{p{6.5cm}}{{\tiny Some notes on table};Some notes on table;Some notes on table;Some notes on table;Some notes on table;Some notes on table;Some notes on table;Some notes on table;}\\ 
\end{tabular}


\subsection{Figure}
\begin{figure}[H]
\caption{Number of venues by delegate capacity and region (2014 Q1)}
\label{fig:001} 
\centering 
  \includegraphics[scale=0.68]{../figures/figure1.pdf}
\end{figure}

\section{Knitr}

\begin{knitrout}
\definecolor{shadecolor}{rgb}{0.969, 0.969, 0.969}\color{fgcolor}\begin{kframe}
\begin{alltt}
\hlkwd{set.seed}\hlstd{(}\hlnum{1213}\hlstd{)}  \hlcom{# for reproducibility}
\hlstd{x} \hlkwb{=} \hlkwd{cumsum}\hlstd{(}\hlkwd{rnorm}\hlstd{(}\hlnum{100}\hlstd{))}
\hlkwd{mean}\hlstd{(x)}  \hlcom{# mean of x}
\end{alltt}
\begin{verbatim}
## [1] -1.94
\end{verbatim}
\begin{alltt}
\hlkwd{plot}\hlstd{(x,} \hlkwc{type} \hlstd{=} \hlstr{"l"}\hlstd{)}  \hlcom{# Brownian motion}
\end{alltt}
\end{kframe}
\includegraphics[width=\maxwidth]{figure/my-label} 

\end{knitrout}



\bibliographystyle{plain}
\bibliography{word_bib}

%This marks the end of the document
\end{document}
